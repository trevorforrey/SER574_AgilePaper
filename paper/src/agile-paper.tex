%
% The first command in your LaTeX source must be the \documentclass command.
\documentclass[sigplan,screen]{acmart}

%
% defining the \BibTeX command - from Oren Patashnik's original BibTeX documentation.
\def\BibTeX{{\rm B\kern-.05em{\sc i\kern-.025em b}\kern-.08emT\kern-.1667em\lower.7ex\hbox{E}\kern-.125emX}}

%
% end of the preamble, start of the body of the document source.
\begin{document}

%
% The "title" command has an optional parameter, allowing the author to define a "short title" to be used in page headers.
\title{Collaborative Work Across Cross Functional Agile Teams}
  
%
% The "author" command and its associated commands are used to define the authors and their affiliations.
% Of note is the shared affiliation of the first two authors, and the "authornote" and "authornotemark" commands
% used to denote shared contribution to the research.
\author{Trevor Forrey}
\authornote{Both authors contributed equally to this research.}
\email{tforrey@asu.edu}
\orcid{1234-5678-9012}
\author{Aman Kaushik}
\authornotemark[1]
\email{akaush13@asu.edu}
\affiliation{%
  \institution{Arizona State University}
  \streetaddress{P.O. Box 1212}
  \city{Mesa}
  \state{Arizona}
  \postcode{85212}
}

%
% The abstract is a short summary of the work to be presented in the article.
\begin{abstract}
'The Agile way' of working has taken over as many organization's number one way to build products.
Although Agile practices have seen a great success, it can still be difficult to get multiple cross-functional agile teams to work together.
Thankfully, there are many strategies to successfully navigate through this network of interlinked agile teams.
Communication, negotiation, challenging assumptions, having a willingness to change, learning from others, team structuring, tooling, and having clear designs are all important for the success of many agile teams.
\end{abstract}

\ccsdesc[500]{Agile Process}
\ccsdesc[300]{Team Success}

%
% Keywords. The author(s) should pick words that accurately describe the work being
% presented. Separate the keywords with commas.
\keywords{agile process, team collaboration, design, architecture, SER574, communication}

\maketitle

\section{Introduction}
Agile has quickly become the go-to way to develop software products.
It allows teams to focus on what's important, like "Individuals and Interactions over Processing and Tools..." \cite{b3}. 
With roles clearly defined, regular communication and feedback, members from different departments - developers, testers, business analysts and product managers can seemlessly work on a single product with the same intent, incentive and vision.
The feedback cycle that the agile process creates also allows customers to be involved in the development of the product, which in turn helps the resulting deliverable.
Most importantly, agile practices allow for changes in requirements without a huge cost to development.
Despite the success and great features of Agile, it can be difficult to productively implement it across teams.
Some common issues with successfully managing a network of cross-functional agile teams are: \cite{b4}
1) The flexibility of agile can cause a lack of communication between teams. 2) Silos can form within the company where each team performs stand ups and communicates, but have no idea what other teams are doing.
3) Bigger organizations tend to have more hierarchy and more rigid culture which can interfere with communication. 4) Teams inside a network may not always be working on the same incentive or vision. 5) Often times, "important" work is not equally distributed among employees which can lead to dependencies being formed on some key employees.
6) Team members may have biases about their work as compared to the work of other teams, which can make communication hard.
Thankfully, there are many strategies to allow multiple Agile teams to still be successful together.
 
\section{Team Structure}
The actual layout of an office can have an effect on the production of a software project.
In traditional businesses, teams are split up based on their role - testers are separated from developers, who are separated from requirements writers.
This role based separation can create silos within a company. 
These silos block team members from naturally collaborating with each other to fix a problem.
While working on a multi-team agile project, it's important to eliminate silos, as it allows for more organic collaboration \cite{b1}.
Even splitting up teams based on their individual agile team will cause silos based on the product features.
For success in multiple agile teams, there must be a balance between keeping team members together and creating a collaborative environment across teams.

\section{Communication}
Communication is one of the most important aspects when working on a multi-team agile project.
Similar to how communication can make or break individual teams, it can have even greater effects on multi-team projects.
When you can't strike a perfect balance in the team layout, alternative forms of communication can help.
Even physical siloes between developers can be overcome by having good communication channels.
These events can come in the form of: cross-team stand-ups, multiple project owners communicating, and cross-team planning \cite{b2}.
By making communication convenient, teams can easily work together to move the project forward.

\section{Clear Goals}
Communication alone will not ensure a successful project, it's important to have a set of clear goals for each team.
These goals should be publicly known, so each team knows exactly what every other team is working on.
With each team having a defined and unique part of the project, they'll be able to work independently, and "...plan for cross-team dependencies" \cite{b2}. 
Similar to convenient communication, having clear goals helps teams organically collaborate.
By knowing what each team is working on, teams with similar goals can make sure they develop something that's concise and can work when combined.

% \section{Cross-Team Events}
% Although convenient communication and clear goals helps, they don't guarentee a sucessful project alone.
% It's important, during multi-team agile projects, to plan for cross-team communication.
% Cross-team events push teams toward collaboration through expressing the state of the project.
% These events 

\section{Tools}
The right tooling can free a multi-team organization to focus on what's most important, development.
Common collaboration tools like Jenkins, GitHub Reviews, and Automated Testing Frameworks, are just a few ways to make developing correct software convenient.
These tools make sure teams don't step on each other's feet by wrongfully adapting another team's features.
Build history from Jenkins can provides visuals to every team on the health of a sub project.
Jenkins also allows for integration tests to be automatically performed, letting developers know if their recent additions have broken a part of the project.
Reviewing code online, automated testing, and many other development tools aid in a less painful way to develop software.
The tooling chosen between teams is important to decide when working on multi-team agile projects.

\section{Agile Architecture}
\subsection{Idea}
The key idea behind an agile architecture and design system/practice is that the existing system should remain operational for all users and at the same time should evolve to cater to new requirements.
The overhead and delays associated with "re-architecturing" a system should be completely eliminated. From an end users' perspective the system should always be in a continuous state of flow \cite{b5}.
The system should be designed for scalability, maintainability, testability and deployability. For example, a micro-services architecture allows for easiser testability and alllows for more decentralization but at the same time it decreases the deployability of the system (as compared to a monolithic architecture).
\subsection{Need for Agile Architecture}
When working on a large system with multiple stakeholders, each stakeholder will have a different investment into the solution and would measure value differently.
Agile architecture plays a significant role in this situation. As its the architech who is responsible to meet each stakeholders' needs, create maximum value for them, maximize utilization of the project resources and for the continuous growth of the system \cite{b6}.
It is necessary that the agile architecture be a hybrid of both "up-front design" to create a reliabile system and "forward-thinking" to meet future needs. This may not necessarily apply to all kind of systems, as a small system can work well only with "upfront design" as well.
Large scale systems on the other hand need both, the bootstrapping nature of "forward-thinking" and the stability and reliability of "upfront-design" to be successful. When working in a network of teams it becomes important to recognise that there will always be a trade off between the two and the same would need to be communicated clearly to all stakeholders to align goals and intent and to maximize collaboration[7].
Keeping in mind that the price of "re-architecturing" a large system may outweigh the value obtained from it, agile architecture should balance the two.
\subsection{Agile Architecture in a Network of Agile Teams}
For successfully creating and maintaining an agile architecture 'style' system, it's important that individual contributors be allowed to provide intputs to the architecture, i.e, its important to create a feedback loop for architecture and design as well and it should not be left upto one key high ranked individual.
Communication is also a key aspect that not only promotes active participation from all stakeholders but also helps in identifying and streamlining the different views, intents and alignments.
Finally, 'change' is at the core of the agile methodology. A good agile architecture should not only plan for change but should plan in a way that will best benefit the customer, the system and the company.   

\section{Conclusion}
The key to making agile work across a network of teams if to infuse 'the agile way' into the system from the ground up.
From the initital architecture to development and testing, every phase should have open communication and a feedback loop, so that every stakeholder can actively participate.
Active communication and feedback also allow for better alignment of of values, goals and intent across the teams and hence result in more productive collaboration.  
 
\begin{thebibliography}{00}
  \bibitem{b1} A. Crocker, R. Cross, H. K. Gardner, "How to Make Sure Agile Teams Can Work Together", Harvard Business Review, 2019. [Online]. Available: https://hbr.org/2018/05/how-to-make-sure-agile-teams-can-work-together.
  \bibitem{b2} "Collaboration Across Agile Teams", GSA Tech, 2019. [Online]. Available: \verb| https://tech.gsa.gov/guides/Collaboration_Across_Agile_Teams|
  \bibitem{b3} "Manifesto for Agile Software Development", 2019. [Online]. Available: http://agilemanifesto.org/.
  \bibitem{b4} https://hbr.org/2018/05/how-to-make-sure-agile-teams-can-work-together 
  \bibitem{b5} https://www.scaledagileframework.com/agile-architecture/
  \bibitem{b6} http://www.agilearchitect.org/agile/principles.htm 
  \bibitem{b7} https://www.ben-morris.com/relax-theres-no-conflict-between-architecture-and-agile/ 
\end{thebibliography}


\end{document}